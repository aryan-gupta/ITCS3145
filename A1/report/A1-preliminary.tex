\documentclass{article}
\usepackage{listings}
\usepackage{pdfpages}
\graphicspath{ {.} }

\usepackage{fancyhdr}
\usepackage[pdftex,
    pdfauthor={Aryan Gupta},
    pdftitle={ITCS3145 A1 Report},
    pdfsubject={Assignment Report},
    pdfkeywords={},
    pdfproducer={Latex with hyperref},
    pdfcreator={}
]{hyperref}
\usepackage[margin=1in]{geometry}
\hypersetup{colorlinks=true, linkcolor=black,urlcolor=black}
\usepackage[margin=1in]{geometry}
\usepackage{lastpage}
\usepackage[margin=1in]{geometry}
\usepackage{fancyhdr}
\usepackage{amsmath,amsfonts, amsthm}
\usepackage{float}
\usepackage{graphicx}

\lstset{
    basicstyle=\ttfamily,
    columns=fullflexible,
    frame=single,
    breaklines=true,
    postbreak=\mbox{$\hookrightarrow$\space},
}

\begin{document}
	\section{A1-preliminary}
	The code for this assignment was very simple. First a char array was created. By reading the man page for \lstinline{gethostname()}, it can be seen that the programmer can use \lstinline{HOST_NAME_MAX} to get the length of the largest host name (not including the null terminator). Then the array is passed into the \lstinline{gethostname()} library function. The code checks for errors and prints the name to stdout. After running the program on mamba, the preliminary\_answer file was created and contained \lstinline{mba-c1.uncc.edu}

\end{document}

